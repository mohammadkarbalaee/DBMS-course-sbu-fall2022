\documentclass[twocolumn,a4paper, 10pt]{article}
\usepackage{fullpage}
\usepackage{titling}

\title{\textbf{Database Management Systems} \\ \emph{1st Assignment}}
\author{Muhammad Karbalaee}

\begin{document}
	\maketitle
    \section*{1st Question}
        Data is every sort of fact about events, people, objects and generally 
        speaking, raw information is called data. \\
        There are four types of data.
        \begin{itemize}
            \item Nominal
            \item Ordinal
            \item Interval
            \item Ratio
        \end{itemize}
        \textbf{Nominal} 


            Nominal data is a type of data that is used to label variables and 
            doesn't provide any quantitative information. \\
            For example; Favorite color of ten people is a nominal data set.
            Red is no greater than blue or you can't calculate the average 
            color that people like. Only calculating the mode of this data set is possible. \\
        \textbf{Ordinal}

        Ordinal data is a type of data that only indicates the order between variables and
        nothing else. \\ For example; being 3rd in comparison with being 2nd 
        has a meaning. Calculating the median of ordinal data is possible.\\ 
        \textbf{Interval}

        Interval data is a data type that preserves the ratio between intervals of variables.
        And absolute zero doesn't have any meaning in this type of data.
        \\ For example; Temperature is a interval data. \\ 
        \textbf{Ratio}

        Ratio data is one that has an absolute zero value and every calculation is 
        possible on ratio data because it not only preserves the ration between intervals 
        , but also it preserves ratio between values themselves. \\ 
        For example; 2 is two times 1. Normal numerical data is ratio data. 

    \section*{2nd Question}
        A database system is a repository that stores and manages data files and records. \\ 
        Generally there are two methods of keeping data records, file systems or databases. \\
        File systems store data records directly into files and reading, writing and updating data 
        must be performed manually by computer programs that programmers create. \\ For example; 
        To set a constraint on the data type that the user of the program can input for the age field,
        the programmer of that software should manually check the input data type and apply the restrictions. \\
        On the other hand, database can satisfy this need easily since databases are basically already-created 
        softwares that have implemented simple-to-use APIs that make the life of programmers easier. \\ 
        Furthermore, databases can reduce redundancy, increase security, maintain integrity, concurrent access, and avoid inconsistencies. \\
        In terms of redundancy, databases manage data records carefully to avoid data records being stored twice unnecessarily. \\
        They insure security by applying access levels which simply means defining which users can access which data and what they can 
        do with it. \\ 
        Integrity is guaranteed by databases through predefined constraints. For example; the database designer can set constraints 
        that an student cannot take more than 20 credits in each semester, this makes the software more reliable and integrated. \\ 
        Sometimes two or more users want to access on data record at the same time. This is known as concurrent access. 
        Databases help developers with that too since they handle all these by themselves. \\ 
        These features make databases the preferred method to store data rather than file systems.
\end{document}